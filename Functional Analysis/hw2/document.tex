
\documentclass[oneside, final, 11pt]{article}
\usepackage[utf8]{inputenc}
\usepackage[T2A]{fontenc}
\usepackage[russian]{babel}

\usepackage{natbib}
\usepackage{graphicx}
\usepackage{amsmath}
\usepackage{xcolor}
\usepackage{amssymb}
\usepackage{graphicx}
\usepackage[document]{ragged2e}
\usepackage[paper=a4paper, margin=2cm, bottom=2cm]{geometry}

\begin{document}
\begin{titlepage}
	
	
	\newgeometry{margin=1cm}
	
	\centerline{\large \bf МИНИСТЕРСТВО ОБРАЗОВАНИЯ РЕСПУБЛИКИ БЕЛАРУСЬ}
	\bigskip
	\bigskip
	\centerline{\large \bf БЕЛОРУССКИЙ ГОСУДАРСТВЕННЫЙ УНИВЕРСИТЕТ}
	\bigskip
	\bigskip
	\centerline{\large \bf ФАКУЛЬТЕТ ПРИКЛАДНОЙ МАТЕМАТИКИ И ИНФОРМАТИКИ}
	\vfill
	\vfill
	\vfill
	\centerline{\Large \bf ФУНКЦИОНАЛЬНЫЙ АНАЛИЗ}
	\bigskip
	\bigskip
	\vfill
	\begin{centering}
		{\large
			Домашняя работа №2\\
			студента 2 курса 2 группы \\}
	\end{centering}
	\centerline{\large \bf Царика Виталия Александровича}
	\vfill
	\vfill
	\hfill
	\begin{minipage}{0.25\textwidth}
		{\large{\bf Преподаватель} \\
			{\it Дайняк Виктор \\ Владимирович}}
	\end{minipage}
	\vfill
	\vfill
	\centerline{\Large \bf Минск 2019}
	
\end{titlepage}

\restoregeometry
\section{162-169}
	Можно ли в нормированном пространстве $\mathbb{R}$ принять за норму элемента $x$:

	\begin{enumerate}
		\item[162.] $f(x) = \sqrt{x}$
		
		$f(x)$ неопределена для всех $x \in \mathbb{R}$
		
		Ответ: нет
		
		\item[163.] $f(x) = \sqrt{|x|}$
				
		$||\lambda x|| = \sqrt{|\lambda x|} = \sqrt{|\lambda|} \sqrt{|x|} = \sqrt{|\lambda|} ||x|| \neq |\lambda|||x||$
		
		не выполнена вторая аксиома
		
		Ответ: нет
		
		\item[164.] $f(x) = |x-1|$
				
		$||0|| = |0 - 1| = 1 \neq 0 $ 
		
		не выполнена первая аксиома
		
		Ответ: нет
		
		\item[165.] $f(x) = \sqrt{x^2}$
		
		$f(x) = \sqrt{x^2} = |x|$
				
		$f(x) = |x|$ - стандартная норма пространства действительных чисел
		
		Ответ: да
		
		\item[166.] $f(x) = 5|x|$
				
		1) $||x|| = 0 \Leftrightarrow 5|x| = 0 \Leftrightarrow x = 0$
		
		2) $||\lambda x|| = 5|\lambda x| = |\lambda|5|x| = |\lambda| ||x||$
		
		3) $||x + y|| = 5|x + y| \leq 5|x| + 5|y| = ||x|| + ||y||$
		
		Ответ: да
		
		\item[167.] $f(x) = x^2$
				
		$||\lambda x|| = (\lambda x)^2 = \lambda^2 x^2 = \lambda^2 ||x|| \neq |\lambda|||x||$
		
		не выполнена вторая аксиома
		
		Ответ: нет
		
		\item[168.] $f(x) = |\arctan{x}|$
				
		$||2 * 1|| = |\arctan{2}| = 0.4142... \neq \frac{\pi}{2} = |2| |\arctan{1}| = |2|||1|| $
		
		не выполнена вторая аксиома
		
		Ответ: нет
		
		\item[169.] $f(x) = \ln{|x|}$
		
		$||1|| = \ln{1} = 0 $
		
		не выполнена первая аксиома
		
		Ответ: нет
		
	\end{enumerate}

\section{170-175}
	Можно ли в нормированном пространстве векторов на плоскости принять за норму элемента $a = (x, y)$:

	\begin{enumerate}
		\item[170.] $f(a) = \sqrt{|xy|}$
		
		$a = (0, 1) $
		
		$||a|| = \sqrt{|0*1|} = 0$
		
		не выполнена первая аксиома
				
		Ответ: нет
		
		\item[171.] $f(a) = |x| + |y|$
		
		1) $||a|| = 0 \Leftrightarrow |x| + |y| = 0 \Leftrightarrow a = (0, 0)$
		
		2) $||\lambda a|| = |\lambda x| + |\lambda y| = |\lambda|(|x| + |y|) = |\lambda| ||a||$
		
		3) $||a + b|| = |x_1 + x_2| + |y_1 + y_2| \leq |x_1|+ |x_2| + |y_1| +|y_2| = ||a|| + ||a||$
		
		Ответ: да
		
		\item[172.] $f(a) = max(|x|, |y|)$
		
		$a = (3, 2), b = (2, 1)$
		
		$||a + b|| = ||(5, 3)|| = 5$
		
		$||a|| + ||b|| = 3 + 2 = 5$
		
		не выполнена третья аксиома
		
		Ответ: нет
		
		\item[173.] $f(a) = \sqrt{x^2 + y^2} + \sqrt{xy}$
		
		Не определена для все векторов. Например, для $a = (1, -1)$
		
		Ответ: нет
		
		\item[174.] $f(a) = |x^2 - y^2|$
		
		$||(1, 1)|| = |1 - 1| = 0$
		
		Не выполнена первая аксиома
		
		Ответ: нет
		
	\end{enumerate}


\section{176-180}
	Можно ли в линейном пространстве дифференцируемых на $[a, b]$ функций принять за норму элемента $x(t)$:

	\begin{enumerate}
		\item[176.] $max_{t \in [a,b]}|x(t)|$
		
		1) $max_{t \in [a,b]}|x(t)| = 0 \Leftrightarrow |x(t)| \leq 0 \Leftrightarrow x(t) = 0$
		
		2) $||\lambda x(t)|| = max_{t \in [a,b]}|\lambda x(t)| = |\lambda| max_{t \in [a,b]}|x(t)| = |\lambda| ||x(t)||$
		
		3) $||x(t) + y(t)|| = max_{t \in [a,b]}|x(t) + y(t)| \leq max_{t \in [a,b]}|x(t)| + max_{t \in [a,b]}|x(y)| = ||x(t)|| + ||y(t)||$
		
		Ответ: да
		
		\item[177.] $max_{t \in [a,b]}|x'(t)|$
		
		$ x(t) = 1$
		
		$ ||x(t)|| = max_{t \in [a,b]}|0| = 0 $
		
		Не выполнена первая теорема
		
		Ответ: нет
		
		\item[178.] $|x(b) - x(a)| + max_{t \in [a,b]}|x'(t)|$
		
		$ x(t) = 1$
		
		$ ||x(t)|| = |1 - 1| + max_{t \in [a,b]}|0| = 0 $
		
		не выполнена первая аксиома
		
		Ответ: нет
		
		\item[179.] $|x(b) - x(a)| + max_{t \in [a,b]}|x'(t)|$
		
		$ x(t) = 1$
		
		$ ||x(t)|| = |1 - 1| + max_{t \in [a,b]}|0| = 0 $
		
		Ответ: нет
		
		\item[180.] $|x(a)| + max_{t \in [a,b]}|x'(t)|$
		
		$||\lambda x(t)|| = |x(a)| + max_{t \in [a,b]}|(\lambda x(t))'| =  |x(a)| + |\lambda| max_{t \in [a,b]}|x(t)'| \neq |\lambda| (|x(a)| + max_{t \in [a,b]}|x(t)'|) = |\lambda| ||x(t)||$
		
		Не выполнена вторая аксиома
		
		Ответ: нет
		
	\end{enumerate}
	

\end{document}